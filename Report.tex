\documentclass[12pt]{article}
\usepackage[utf8]{inputenc}
\usepackage{amsmath}
\usepackage{graphicx}
\usepackage{hyperref}
\usepackage{geometry}
\usepackage{tocloft}
\usepackage{times} % Use Times New Roman font
\geometry{a4paper, margin=1in}

\title{Architecture of Peer-to-Peer Systems}
\author{Tran Manh Duy}
\date{December 23, 2024}

\begin{document}

\maketitle

\tableofcontents
\newpage

\section*{Abstract}
Peer-to-Peer (P2P) architectures enable decentralized resource sharing and file transfer directly between nodes without the need for central authority. This report analyzes the P2P system architecture, highlighting types, components, routing mechanisms, scalability, and applications. The discussion also emphasizes the evolution of P2P systems and their potential in emerging technologies like blockchain and IoT.

\section{Introduction}
Peer-to-Peer (P2P) systems play a significant role in modern distributed computing by enabling direct communication and resource sharing between nodes without requiring a centralized server. Their importance spans various domains, from file-sharing applications to decentralized blockchain networks.

The emergence of P2P systems was driven by the need for scalability and resilience in data exchange networks. Over the years, P2P technology has evolved, integrating advanced routing mechanisms and structured overlays to address challenges like performance optimization and fault tolerance. By eliminating central points of failure, P2P systems enhance robustness and foster a more democratic approach to resource sharing.

This report delves into the architectural types, core components, routing techniques, scalability, and real-world applications of P2P systems, underscoring their critical role in enabling efficient and resilient distributed systems. The discussion also explores challenges such as security, consistency, and resource allocation, providing insights into the ongoing advancements in P2P networks.

\section{Types of P2P Architectures}

\subsection{Centralized P2P}
Centralized P2P systems depend on a single server to index resources and facilitate peer connections. This model simplifies the search process and offers quick resource discovery, as exemplified by Napster. However, centralized systems are vulnerable to single points of failure, raising concerns about reliability and legal risks.

Despite their limitations, centralized P2P systems laid the groundwork for modern decentralized models by popularizing the concept of file-sharing among large user bases. Central servers also act as a control point, ensuring a level of monitoring and management that unstructured systems may lack. Examples include early online music-sharing platforms that facilitated wide-scale adoption of P2P technologies.

\subsection{Decentralized Unstructured P2P}
Unstructured P2P systems operate without predefined network topology, allowing nodes to broadcast queries to neighbors for resource discovery. While systems like Gnutella and Kazaa highlight the simplicity of this approach, they also underscore its inefficiency due to excessive network traffic during searches.

These systems are highly fault-tolerant, as they do not rely on any specific node. However, their lack of scalability limits their application in large-scale networks requiring high performance. Enhancements like supernodes have been introduced to mitigate these limitations by acting as hubs for query routing. Modern implementations have also explored hybrid designs, combining elements of structured and unstructured models to optimize efficiency.

\subsection{Decentralized Structured P2P}
Structured P2P systems use Distributed Hash Tables (DHTs) to organize nodes and data systematically. This ensures efficient resource lookup, making them suitable for large-scale applications like BitTorrent and Kademlia. Structured systems map resources to specific nodes based on hash values, enabling predictable search times and efficient data distribution.

By implementing DHTs, structured systems offer a balance between efficiency and scalability. However, they require more complex algorithms for implementation and maintenance. Advanced features like dynamic load balancing and fault recovery are integral to maintaining structured P2P networks' efficiency. Innovations in adaptive DHT designs and multi-layer overlays further enhance their utility in handling diverse workloads.

\section{Key Components of P2P Architecture}
\begin{itemize}
    \item \textbf{Peers (Nodes):} Each participant in the network functions as both a client and a server, contributing to resource sharing and data exchange. Nodes may have varying capabilities, ranging from personal devices to powerful servers. Dynamic peer behavior, including node churn, poses challenges in maintaining stability and consistency.
    \item \textbf{Overlay Network:} The logical structure connecting peers determines routing efficiency and network performance. Overlay types include ring, mesh, and hierarchical topologies. The design of the overlay network significantly impacts the network's scalability and fault tolerance. Advanced overlays integrate fault-tolerant mechanisms and self-healing properties to maintain robustness.
    \item \textbf{Routing and Lookup:} Mechanisms like DHTs and flooding ensure effective resource discovery and data transfer between peers. Efficient routing minimizes latency and enhances scalability. Lookup protocols must handle dynamic changes in the network, such as node churn, to maintain reliability. Techniques like proximity-aware routing further optimize network performance.
\end{itemize}

\section{Peer-to-Peer Routing}

\subsection{Flooding (Unstructured)}
Flooding is a straightforward routing method where nodes broadcast queries to all their neighbors. This approach ensures that resources are located, but it generates excessive network traffic and scales poorly.

To improve efficiency, techniques such as Time-To-Live (TTL) and query limits are often employed. These strategies reduce redundant transmissions while maintaining high availability. Flooding remains a practical approach for small networks or where query latency is not a critical concern. Recent studies suggest hybrid flooding mechanisms that combine controlled broadcast with selective query forwarding to enhance scalability.

\subsection{DHT-Based Routing (Structured)}
DHT-based routing maps keys to specific nodes using hash functions, enabling deterministic and efficient resource discovery. Systems like Chord employ circular DHTs, while Kademlia uses XOR-based distance metrics for routing.

These methods enhance scalability by ensuring logarithmic time complexity for resource lookups. However, maintaining DHT consistency requires sophisticated algorithms to handle dynamic node join/leave events. The use of replication and caching further improves the performance and fault tolerance of DHT-based systems. Innovations in decentralized load balancing and proximity-aware replication add further robustness.

\section{Scalability and Maintenance}

\subsection{Node Joining and Leaving}
Structured P2P systems dynamically adapt to changes by redistributing resources and updating routing tables. This minimizes disruption and maintains overall network efficiency. Protocols like Chord and Pastry implement mechanisms to reassign node responsibilities seamlessly during such transitions.

Unstructured systems, while less organized, handle churn rates effectively due to their redundant connections, albeit at the cost of higher overhead. Techniques like neighbor caching and backup paths enhance resilience against frequent node departures. Research into adaptive algorithms for churn management aims to further optimize these systems.

\subsection{Replication}
Replication ensures data availability and fault tolerance by duplicating resources across multiple nodes. This strategy is crucial for mitigating the impact of node failures or network disruptions.

Replication policies, such as full and partial replication, balance storage costs and performance based on application requirements. Additionally, adaptive replication schemes respond to network conditions, ensuring optimal resource distribution and availability. Real-time replication techniques enhance the reliability of mission-critical applications.

\section{Real-World Applications of P2P Architectures}

\subsection{Distributed Storage Systems}
Platforms like IPFS leverage structured P2P systems to create decentralized storage networks. This approach eliminates dependency on centralized servers, enhancing fault tolerance and reducing costs. IPFS allows users to store and share content-addressable data efficiently, enabling applications in archival storage and data distribution.

By utilizing DHTs, these systems enable efficient content retrieval, making them suitable for hosting large-scale, immutable data repositories. Use cases include scientific data sharing, media distribution, and censorship-resistant content hosting. Advanced implementations in IPFS focus on improving data deduplication and retrieval speeds.

\subsection{Blockchain Networks}
Blockchain platforms like Ethereum and Bitcoin use P2P networks to maintain ledger consistency and support decentralized consensus mechanisms. Structured overlays ensure scalability and prevent single points of failure.

P2P architectures enable trustless interactions between participants, fostering the development of decentralized finance (DeFi) and other blockchain-based applications. Smart contracts further extend the functionality of these platforms, supporting automated and secure digital agreements. Emerging applications include non-fungible tokens (NFTs) and decentralized identity systems.

\subsection{Content Delivery Networks (CDNs)}
Peer-assisted CDNs reduce server loads and latency by distributing content delivery responsibilities among peers. This model enhances scalability and ensures faster access to data for end-users.

Streaming platforms like Netflix and YouTube use P2P technologies to optimize bandwidth usage during peak times, demonstrating the practical benefits of decentralized delivery mechanisms. Emerging technologies like WebRTC further expand P2P capabilities in real-time multimedia communication. Enhanced algorithms for adaptive streaming improve quality of experience in dynamic network environments.

\subsection{Online Gaming}
Multiplayer online games employ P2P systems for real-time communication and resource sharing. This reduces server dependency and ensures seamless gameplay experiences for players worldwide.

Techniques like proximity-based matchmaking optimize P2P connections, enhancing the overall quality of service in gaming environments. Additionally, hybrid P2P-client-server models balance scalability and control, providing reliable and immersive gaming experiences. Innovations in latency-reduction techniques further improve real-time responsiveness in competitive gaming scenarios.

\section{Conclusion}
P2P architectures have revolutionized the way resources are shared and accessed in distributed systems. Their decentralized nature ensures scalability, fault tolerance, and efficiency, making them indispensable in modern computing.

The future of P2P technology lies in its integration with emerging fields like IoT, AI, and edge computing, promising enhanced capabilities and broader applications. By addressing existing challenges, such as security and scalability, P2P systems are poised to remain at the forefront of technological innovation. Research into adaptive routing, secure communication protocols, and efficient data distribution will further strengthen their role in diverse domains.

\newpage
\begin{thebibliography}{99}

\bibitem{napster} Napster: The First Peer-to-Peer File Sharing Service. Available at: \url{https://example.com/napster}.

\bibitem{gnutella} Gnutella Protocol Specification. Available at: \url{https://example.com/gnutella}.

\bibitem{chord} Stoica, I., Morris, R., Karger, D., Kaashoek, M. F., \& Balakrishnan, H. (2001). Chord: A scalable peer-to-peer lookup protocol for internet applications. Proceedings of the 2001 conference on Applications, technologies, architectures, and protocols for computer communications.

\bibitem{ipfs} IPFS: The InterPlanetary File System. Available at: \url{https://ipfs.io/}.

\bibitem{blockchain} Nakamoto, S. (2008). Bitcoin: A Peer-to-Peer Electronic Cash System. Available at: \url{https://bitcoin.org/bitcoin.pdf}.

\bibitem{webrtc} WebRTC: Enabling Real-Time Communications in Browsers. Available at: \url{https://webrtc.org/}.

\bibitem{p2p_networks} P2P Networks and Blockchain: Overview and Innovations. Available at: \url{https://example.com/blockchain_p2p}.

\end{thebibliography}


\end{document}

