\section{Application of P2P}

\subsection{Advantages and disadvantages of P2P}

\textbf{Advantages}
\begin{itemize}
    \item P2P networks do not require the use of servers.
    \item Each computer device is a separate user manager.
    \item P2P operations do not require any complex specialized knowledge.
    \item Home and small business environments are suitable for using P2P networks.
    \item Does not require too much traffic when accessing the network.
\end{itemize}

\textbf{Disadvantages}
\begin{itemize}
    \item Information on the device cannot be backed up centrally.
    \item If multiple computer devices access at the same time, it will reduce performance.
    \item Data sets are not scientifically arranged but are stored on personal computers. This significantly affects the process of determining their location.
    \item Only providing some basic rights, poor security.
\end{itemize}

\subsection{Real-World Applications of P2P}

\begin{enumerate}
    \item File sharing: P2P is widely used to share files such as music files, videos, etc. The most typical application is Spotify's application so that users can search and save their favorite music. In more detail, Spotify has applied P2P network structure.
    
    \item Cryptocurrency: The basis of prominent cryptocurrencies such as Bitcoin is blockchain technology. Transactions all use the P2P network to authenticate transactions and maintain some decentralization. P2P technology ensures equality and transparency in financial transactions.
    
    \item Internet of Things (IoT): P2P is a technology that communicates directly between devices without relying on a central server. Therefore, P2P allows for faster data transmission and reduced latency in IoT applications.
\end{enumerate}

\subsection{Future Applications of P2P}

\begin{enumerate}
    \item Data sharing and privacy: P2P will improve and develop more in terms of security, while strengthening data control. This helps individuals manage their assets more effectively.
    
    \item Edge computing: P2P has the potential in edge computing, allowing devices at the edge of the network to communicate and collaborate effectively. P2P technology can improve the performance and scalability of edge computing applications by distributing computing resources and data processing.
    
    \item Collaborative Workspace: P2P technology can create a private network within the workspace by allowing secure and efficient resource sharing among distributed teams. Create a secure and highly productive collaborative workspace.
\end{enumerate}
